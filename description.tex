\centerline{\bf I. Introduction (0.5p)} \smallskip

The field of time--domain astrophysics is currently experiencing a renaissance,
driven by the fusion of large datasets, computational infrastructure, and
astro--statistical tools.  Research efforts may be broadly cast into two types
of analyses, data--mining efforts (discovery) and follow--up of interesting
events (characterization).  The most ambitious of these efforts utilize
real--time discovery tools on survey data streams to drive autonomous follow--up
resources. As these data streams become increasingly more vast and complex, the
statistical tools required to efficiently drive these resources must themselves
increase in complexity.  They must be sensitive to all quadrants of the
Rumsfeldian "known" vs. "unknown" characterization scheme, with the majority of
phenomena falling in the "known known" category, and the most interesting of
events falling within the "unknown unknown" category.  To optimally do this
sifting, the complexity of the models much match the complexity of the data
stream to allow a single classification assessment.

In this proposal, we will expand upon the current state--of--the art in event
classification to generate astrophysical variability models with the same
dimensionality as the data being collected. Next generation time--domain data
streams will provide constraints on a given event at irregular times, and in one
of several passbands.  The event models themselves must therefore be inherently
temporal {\it and} spectral, to federate the ensemble of survey and follow--up
data into a single statistical assessment of event type.  The building of these
models requires adopting a unifying description of astrophysical variability.
{\bf We propose sets of spectral--temporal surfaces, generated for all
astrophysical event types, as this optimal description of variability.}  The
process will incorporate the diversity of existing data into a single
statistical model representing the mean behavior of each event class, as well as
higher--order moments about this mean that reflect the intrinsic dimensionality
of the phenomena.  These models will be optimal in the sense that they
incorporate extant multi--passband knowledge (photometric {\it and}
spectroscopic) into an empirically--derived event model.  They will be optimal
in these sense that they represent each class of phenomena at all times and
integrable over any photometric passband, a level of complexity needed to
evaluate the event streams of next--generation surveys such as LSST. And they
will be optimal in the sense that they will unify the statistical description of
astronomical variability from today's heterogeneous standards into a common
language.  To provide the broadest possible impact from this effort, we will
build a classification infrastructure incorporating these models into a
real--time event broker that will be available to the astronomical community.


\medskip {\centerline{\ub{\sc Time--Domain Astronomy (1p)}}} \smallskip

The field of time--domain astronomy began in earnest in the 1990s with
wide--field (10 square degree) microlensing surveys such as MACHO
\citep{2000ApJ...542..281A}, OGLE \citep{1994AcA....44..227U}, and EROS
\citep{2003A&A...400..951A}.   Even then, team members recognized that there was
value in responding to the real--time status of on--going events, which might
reveal exotic effects such as parallax, resolution of the lensed star's disk, or
lens or source binarity.  Team members designed real--time alert streams that
fed directly into follow--up networks such as GMAN \citep{2000PhDT.......258B}
and PLANET \citep{1998ApJ...509..687A}, which resolved for the first time many
of the exotica expected of complex gravitational lens systems.

Subsequent time--domain efforts turned their focus to precision cosmology
measurements through surveys for Type Ia supernovae \citep{1996AJ....112.2398H},
which most prominently yielded the discovery of the acceleration of the
expansion of the Universe \citep{1998AJ....116.1009R,1999ApJ...517..565P}. These
surveys also required real--time analysis of their data to acquire spectroscopic
redshifts of the events. In these cases, the scientists needed to not only know
that the event was happening, but have an estimate of the event type, so as not
to waste precious follow--up resources on non--Ia events.  Contextual clues were
used to assist in event classification, including proximity to host galaxy and
host galaxy type.

Gamma-ray bursts, intense but evanescent flashes from exploding and colliding
stars, are observable across the electromagnetic spectrum but only if
discovered, disseminated, and followed up rapidly. Intense multi-color imaging
coupled with synoptic spectroscopy in the minutes to hours after events are the
state-of-the-art in the field. The most precious (scientifically) tend to be
those events from the lowest and highest redshift, providing vistas on star
formation, dust obscuration, and potentially serving as the bridge between the
electromagnetic and gravity wave landscapes. However, at a discovery rate of
$\sim$100/yr (primarily from NASA/Swift), the community cannot adequately
followup all events with the available resources for target of opportunity
observations. Knowing a priori which GRBs are likely to have the largest return
given substantial telescope investment is crucial. We (Morgan et al., submitted)
have begun to using machine learning tools to try to predict high-redshift
events from immediately available burst data, couching the prediction as a
resource maximization problem.

%morgan: http://qmorgan.org.org/rategrbz/paper.html

In the early 2000s surveys began to categorize and release {\it all} types of
events found in their data, in near real--time. This includes the Deep Lens
Survey \citep{Becker04a} and the Faint Sky Variability Survey
\citep{2003MNRAS.339..427G}.  In these cases, coarse attempts were made at event
classification for all objects displaying astrometric or photometric
variability. Primarily contextual information were used (e.g. ecliptic latitude,
distance from host galaxy) and classifications were done by humans at the
telescope or remotely using web--based visual classification tools.  More recent
time--domain surveys such as the Palomar--Quest survey
\citep{2008AN....329..263D} and Catalina Real-Time Transient Survey
\citep{2011arXiv1102.5004D} have begun to adopt modern statistical techniques to
classify events.  This includes morphological classification of the pixel--level
flux that triggered the event \citep{2008AIPC.1082..252D}, as well as contextual
information regarding the location of the variability
\citep{2010ASPC..434..115M}.  We note that colors of the quiescent objects are
used in these classification schemes, but {\it not} the time--varying color of
the variable flux.

%Kepler classification? \citep{2010ApJ...713L.204B}

In the NSF-sponsored Palomar Transient Factory, we have begun to use machine
learning techniques to discover new events through image differencing,
identifying roughly 1000 variable stars and transients out of 1.5 million
candidates per night. These events are, in real-time, classified with an ML
classifier that makes use of contextual information (such as color of the
nearest galaxy) and temporal metrics (such as difference in magnitude between
the event and the quiescent counterpart brightness). We have shown
\citep{2011arXiv1106.5491B} the ability to  distinguish between transients and
variable stars with a 3.8\% overall error rate (with 1.7\% errors for imaging
within the Sloan Digital Sky Survey footprint). At $>$96\% classification
efficiency, the samples achieve 90\% purity. Determining just what sort of
transient is found (e.g. supernova, nova, QSO event) has proven more
challenging. Only with retrospective analysis, once sufficient data on a light
curve has been obtained, that the classification errors begin to drop
significantly \citep{2011ApJ...733...10R} (see \citealt{2011arXiv1104.3142B} for
a review).

% rs http://arxiv.org/abs/1101.1959 % rb http://arxiv.org/abs/1104.3142

% bloom http://arxiv.org/abs/1106.5491


In all surveys mentioned above, the volume of data being collected made the
surveys sensitive to new types of astrophysical phenomena, with a rapid response
component that enabled immediate and early study.  It is also the case that many
of these efforts designed their event filters based upon the particular survey
design they were being applied to.  To continue this trend into the future,
where surveys such as Gaia\footnote{http://gaia.esa.int} and
LSST\footnote{http://www.lsst.org} will report on {\it all} classes of
variability, a class of models needs to be generated that match the ambitious
designs of the surveys (massive data rates, sparse sampling in time,
inhomogeneous sampling in wavelength).  This is particularly important for
driving the next generation of autonomous follow--up resources, such as LCOGT
\citep{2008AN....329..269H}, that must algorithmically sift through the data
streams to achieve their particular science goals.

\medskip {\centerline{\ub{\sc Strides Towards Spectral--Temporal Event
Classification (1.5p)}}} \smallskip

While teams such as the Harvard Time Series Center \citep{} and the Berkeley
Transients Classification Pipeline \citep{} have made substantial advances in
automated event classification, their methods remain inherently
one--dimensional. That is, their models of event behavior are relevant for data
in a single passband, and do not take into account color evolution during the
events.

One field where there has been substantial progress in spectral--temporal
modeling is in the field of supernova studies. The pioneering work of
\cite{2002PASP..114..803N} integrated a large, inhomogeneous sample of
spectroscopy of Ia supernovae into a single representation of the Ia phenomena.
This integrated model represented the behavior of a typical "Branch--normal"
Type Ia supernova, and defined the spectrum of a typical event at all integer
wavelengths between 1000~\AA~and 25000~\AA, 1 per day for 90 days after
explosion.  Nugent subsequently expanded his template set to include separate
models for intrinsically bright and faint Ia, supernovae Type Ib/c, and Types II
P/L/n. Such templates played a crucial role in real--time event classification
for the SDSS-II Supernova Survey \citep{2008AJ....135..338F}, and enabled a
$90\%$ targeting efficiency for Type Ia supernova \citep{2008AJ....135..348S}.
However, since they only existed for supernova--class events, all events that
did not conform to the spectral and temporal behavior of these models were
subsequently ignored.

This notion of spectral--temporal event representations was expanded upon  by
\cite{2007A&A...466...11G} in the generation of their SALT--II supernova model
(see also \cite{2007ApJ...663.1187H}).  This model incorporated spectroscopic
{\it and} photometric data from low-- and high--redshift supernovae to yield a
set of lightcurve templates describing the behavior of {\it all} Type Ia
supernovae. The spectroscopic data constrained the templates coarsely in time
but finely in wavelength, while the photometric data constrained the templates
more densely in time, but coarsely in wavelength.  The photometric data were
also able to be calibrated, in an absolute sense, to a much higher accuracy than
the spectroscopic data, and were used to bootstrap the overall calibration of
the model.  These data in combination yielded a highly constrained
two--dimensional surface in time and wavelength that describes the temporal
evolution of the rest-frame spectral energy distribution (SED) for SNe Ia.
Importantly, the SALT--II model yields not just the average behavior of the Type
Ia sample, equivalent to the \cite{2002PASP..114..803N} data, but also includes
a first "principal component" about the mean, which captures the
spectral--temporal behavior of the diversity (intrinsically bright to
intrinsically faint) of Ia. This single model -- mean surface plus a first
moment -- fully captures the diversity of behaviors and correlations seen in the
Ia population such as the $\delta m_{15}$ brightness--decline
\cite{1993ApJ...413L.105P} or brightness--stretch \citep{1998AJ....116.1009R}
relations.  The need for only a single principal component reflects the fact
that Ia supernova are intrinsically a one--parameter family; the SALT--II model
captures this in a spectral--temporal representation.

Figure~\ref{fig:salt2} displays the mean surface ({\it left}) and first
principal component surface ({\it right}) from the SALT--II model.  The behavior
for a given Ia lightcurve may be generated by an overall scaling of the mean
surface (representing distance modulus) with a contribution from the secondary
surface representing its place within the 1--parameter family of Ia supernovae.
Importantly, an incoming data stream can be compared to these surfaces, and a
statistical assessment made whether or not it is behaving in accordance with
this model.  This likelihood may be generated using the surfaces, the
photometric uncertainty on the incoming data, and (optionally) priors on the
amplitude of the second surface indicated by the best--fit.  This is the process
we envision resulting from this proposal : {\bf an incoming event stream will be
evaluated against an ensemble of these surfaces to yield a statistical
likelihood that the event is of each class.}


\begin{figure}[t]
\centerline{\psfig{figure=figures/surface0.eps,width=0.5\textwidth} \hfil
\psfig{figure=figures/surface1.eps,width=0.5\textwidth}} \smallskip
\caption[]{\footnotesize Spectral--temporal surfaces of Type Ia supernovae from
the SALT--II model of \cite{2007A&A...466...11G}.  Surfaces are generated from
the aggregation of photometric and spectroscopic data on several hundred vetted
Ia supernova.  The leftmost figure represents the mean behavior of the sample,
defined every 10~\AA~between 2000~\AA~and 9200~\AA, and in daily intervals from
-20 days from B--band peak brightness to +50 days.  The right figure displays
the first moment of these data about the mean distribution, derived using
principal component analysis.  The diversity of Ia lightcurves (across all
passbands and at all times) can be represented by a scaled addition of this
secondary surface to the primary surface, reflecting the observation that Ia are
intrinsically a 1--parameter family.  {\bf We propose here to make similar
models for {\it all} classes of astronomical variability, which will serve as
the common currency in general event classification efforts.}} \medskip \hrule
\label{fig:salt2} \end{figure}

\bigskip \centerline{\bf II. Proposed Work} \smallskip

We propose to make spectral--temporal (ST) surfaces for {\it all} major classes
of astronomical variability.  The individual components of this effort include:

\begin{itemize}

\item the aggregation of photometric and photometrically--calibrated
spectroscopic data from a diverse range of input sources;

\item the generation of a spectral--temporal variability model from all data on
a given class;

\item an investigation into the dimensionality of each class, to determine how
many principal components to include in the model;

\item a statistical framework to generate likelihoods that an incoming event
stream is consistent with each variability model;

\item and an on--line classification service that includes our
spectral--temporal models as part of an overall (single--epoch and multi--epoch)
event assessment.

\end{itemize}

These surfaces will serve as common currency in future event classification
efforts, alongside extant efforts such as: Artificial Neural Network
\citep{2008AN....329..263D}, Support Vector Machine \citep{2007ApJ...665.1246B},
and human--vetted \citep{2011arXiv1106.5491B} classifiers of pixel--level
artifacts; contextual understanding of new single--epoch events using Markov
Logic Networks \citep{2011arXiv1110.4655D} and feature--based classifiers
\citep{2011arXiv1106.5491B}; and shape--based Random Forest
\citep{2011ApJ...733...10R} classifiers of lightcurves in a single passband. Our
effort will add an extra dimensionality to these state--of--the--art
classification efforts, allowing the incorporation of information at multiple
epochs and wavelengths into the overall classification scheme.  Below we
describe in detail each step in this process.


\medskip {\centerline{\ub{\sc Aggregation of Data (2p)}}} \smallskip

The first step in this process is to aggregate the diversity of time--domain
data above phenomena.  Since we will be incorporating data taken on different
instruments, in different filters, and at different epochs, we must enforce a
strict set of standards to ensure we can merge these data into a single ST
model. In practice, this will limit our sources of data to archival resources
where calibration metadata is included in the data curation.

All incoming photometric data {\it must} be accompanied by a filter profile
corresponding to the transmission profile of the filter of observation, as well
as a time stamp corresponding to the epoch of observation.  We will convert each
filter profile into an overall atmosphere plus system throughput representing
the dimensionless probability that a photon of wavelength $\lambda$ reaches the
detector.  This filter profile will be used to weight each data point's
contribution to the ST surface.  We will also use each filter profile to convert
(if necessary) the flux into the AB magnitude system.  We will convert each
input time stamp to correspond to the mid--point of observation, in Barycentric
Julian Date in the Barycentric Dynamical Time standard
\citep{2010PASP..122..935E}.

All incoming spectroscopic data will have the same requirement on epoch of
observation, with the additional requirement that it has been
spectrophotometrically calibrated.  This is difficult.

An additional requirement for objects at cosmological distances is that any
photometric data must be accompanied by a redshift to translate the
observer--frame filter profile into a rest--frame window. This has the
disadvantage that K--correction of the photometry \citep{2002astro.ph.10394H}
requires an instantaneous spectrum of the event, which leads to a
chicken--and--egg problem since that exact spectrum is expected to result from
the ST modeling.  This will require a boot--strapping of the cosmological
K--corrections with an initial ST surface.  However, these K--corrections will
allow us to build ST surfaces that are well--calibrated in the u--band, where
ground--based calibration is difficult and has in the past lead to systematics
in event models \citep{2009ApJS..185...32K}, by using photometry of objects at
moderate redshift.

Identified input data sources include:

\begin{itemize}

% Variable stars

\item SDSS Stripe--82 where we will use the six {\it ugriz} per--camera column
filter profiles for SDSS \citep{2007AJ....134..973I}.

\item Variable stars from the OGLE and Hipparcos surveys cataloged by
\cite{2007A&A...475.1159D}.

% Supernova

\item SDSS Stripe--82 supernova

\item Carnegie SN project

\item The on--line supernova spectrum archive

\item Saurabh's data

\item Time--domain spectroscopy projects TDSS if it happens.

% Transients

\item PTF

\end{itemize}


\medskip {\centerline{\ub{\sc Generation of Spectral--Temporal Surfaces (2p)}}}
\smallskip

We proceed using the following underlying model for astrophysical variability:
the flux from an astronomical object may be represented as $F_\nu(\lambda, t)$
with $F_\nu$ the specific flux of an object above the atmosphere, in Janskys (1
Jy = $10^{-23}$ erg cm$^{-2}$ s$^{-1}$ Hz$^{-1}$), as a function of wavelength
and time.  Our spectral--temporal models will be defined over the grid of
$\lambda$ and $t$, with the bin sizes dependent on the amount of input data
available.  These models will be defined in the event rest--frame, and (to the
best of our ability) above the Earths's atmosphere, meaning the input data will
need to be calibrated in an absolute sense. Data on a single event will only
coarsely sample this surface.  However, the phase space will be more densely
sampled by an ensemble of data on different events of the same class.


Between emission and measurement, $F_\nu$ is attenuated by a (typically unknown)
atmospheric transmission function $T^{atm}(\lambda, t)$, the probability that a
photon of wavelength $\lambda$ successfully propagates through the atmosphere,
and a (typically measured, and occasionally well--measured) system transmission
probability per unit photon $T^{sys}_{filt}(\lambda, t)$.  This latter term
should include the wavelength dependence of the mirror reflectivity, lens and
filter transmission, and detection sensitivity.  The product of these two
defines the overall transmission profile of a given observation in a given
filter $T_{filt}(\lambda, t)$.

Many surveys provide only measurements of $T^{sys}_{filt}(\lambda, t)$
\citep[e.g.][]{2006ApJ...646.1436S} meaning we must synthesize the atmospheric
transmission function for a given observation.  This is a difficult task, since
the atmospheric components that contribute to $T^{atm}(\lambda, t)$ (including
water vapor, aerosol scattering, Rayleigh scattering and molecular absorption)
may vary on the order of $10\%$ per hour \citep{2007PASP..119.1163S}.  So as to
not fall down the rabbit hole in terms of atmospheric calibration, we will adopt
fiducial MODTRAN atmospheres \citep{1999SPIE.3756..348B} at the airmass of
observation (when reported) or adopt a fiducial airmass of 1.3 when not.  We
note that surveys such as SDSS define their filter profiles with the atmospheric
component rolled in \citep{2007AJ....134..973I}.

The total counts transmitted through the optical system are: $$F^{obs}_{filt}
(t) = \int F_\nu(\lambda, t) ~~ T_{filt} (\lambda, t) ~~ \lambda^{-1} d\lambda$$
where $T_{filt}(\lambda, t)$ is the overall transmission probability per unit
photon \footnote{The term $\lambda^{-1}$ comes from the conversion of energy per
unit frequency into the number of photons per unit wavelength}. To place the
final flux measurement on the AB magnitude system, this integral is normalized
by $F_{AB} = 3631$ Jy.

From the available data on each input training event, we will create a sparse
surface labelled $f(\lambda, t)$ that constrains the overall model.  Spectral
data will be rebinned to the wavelength resolution of our model, and photometric
flux will added to the model across wavelengths with a weighting of
$T_{filt}(\lambda, t)~\lambda^{-1}$.  To generate the ST model from these data,
we proceed with the standard assumption that each dataset $f_0(\lambda, t),
f_1(\lambda, t)~\ldots~f_i(\lambda, t)$ may be recovered from a linear
combination of (as--yet unknown) basis surfaces $$f_i(\lambda, t) = \sum_{j=0}
x_{ij} \times {\bf S_j}$$ where $\bf S_j$ are said basis surfaces and $x_{ij}$
are their weightings, which will be different for each given event.  By choosing
the right basis, we may approximate each event with a finite number of surfaces
$\bf S_j$. Principal Component Analysis (PCA) is an optimal way to select this
basis from a ensemble of input data \cite[e.g.][for astrophysical application to
galaxy spectra]{1995AJ....110.1071C}, and "sparse" or "gappy" PCA is a
particular implementation to be used when the inputs sample only a small
fraction of the model \citep{zouht04}, such as here.  As is standard when using
PCA decomposition, we will first create a "mean" surface ${\bf S_M}$ and find
the principal variations about this mean: $$f_i(\lambda, t) = {\bf S_M} +
\sum_{j=1} x_j \times {\bf S_j}.$$ We emphasize that these basis surfaces (${\bf
S_M}$,${\bf S_1}$,${\bf S_2}$ $\ldots$ ${\bf S_j}$) are the fundamental products
of this proposal.

Algorithmically, this process is as follows: accumulate data on each instance of
the event type to be modelled; for each instance, create a sparse data surface
$f_i(\lambda, t)$; from the ensemble of $f_i(\lambda, t)$, find the mean ${\bf
S_M}$; subtract the mean ${\bf S_M}$ from all $f_i(\lambda, t)$; run a
Karhunen--Loeve decomposition \citep{Karhunen:47,Loeve:48} on the
mean--subtracted $f_i(\lambda, t)$, yielding principal surfaces ${\bf S_j}$;
find the coefficients $x_{ij}$ for each $f_i(\lambda, t)$ through the dot
product $x_{ij} = f_i(\lambda, t) \cdot {\bf S_j}$.

\smallskip {\bf Intrinsic Dimensionality of each Basis:} The Principal Component
Analysis is fundamentally an eigenvector analysis, with eigenvalues that
represent the variance in the population that is represented by each surface.
Surfaces that represent effects that dominate the behavior of the population
(e.g. the brightness--decline for Supernova Ia) will have large eigenvalues,
with secondary effects less important and thus having smaller corresponding
eigenvalues in the PCA.  One can thus derive the intrinsic dimensionality of the
population from the spectrum of eigenvalues. Correspondingly, one can well
approximate any event within this population by using the first few bases sorted
by eigenvalue.  We will use such a truncated expansion when creating our basis
models; only surfaces representing the top $90\%$ (or $95\%$, or $99\%$,
depending on the amount of input data) will be retained in the event modeling
process described above.

\smallskip {\bf Uncertainties on the Surfaces:} We with to capture excess
variance in the models that is not represented by the surfaces, e.g. portions of
large model uncertainty that we will want to deweight when fitting.  This excess
variance will be modelled using a jackknife resampling procedure: for each input
event, we will remove it from the training sample and re--create the model.  The
variance of these residuals across all jackknife resamplings $V(\lambda, t)$
will be used as an empirical estimate of the excess variance about our model.

\smallskip {\bf Variants to this Procedure:} We recognize that for certain
phenomena, this basic derivation of the surfaces is not sufficient to completely
describe the events. For example, objects with periodic variability require
folding at the correct period before fitting to these surfaces.  We must
therefore construct periodograms for these lightcurves, optimally using the
ensemble of event data to yield a single period assessment.  This will require
development of multi--band period assessment tools.  Two options we will pursue
are: the generation of periodograms for each passband of information, the
conversion of these periodograms into likelihood functions using a false--alarm
probability analysis \citep[e.g.][and references therein]{2009A&A...496..577Z},
and using the product of these likelihood functions in a final period
assessment; or to use priors on color--evolution derived from our ST surfaces to
wrap all photometric data into a periodogram analysis.  Other classes of events
such as supernovae undergo substantial extinction from their host galaxy, which
will create a color--dependent non--intrinsic reddending of the underlying ST
surface.  This will require an assessment of the reddening of each input vector,
and an additional fit parameter on each incoming lightcurve representing the
degree of extinction \cite[Equation 1][]{2007A&A...466...11G}.

\smallskip {\bf Previous Work:} We note that this procedure follows very closely
the SALT--II model of \cite{2007A&A...466...11G}.  In the SALT--II model, the
degree to which the process captured the behavior of the training data is
represented in the RMS dispersion of the Ia distance moduli in the Hubble
diagram of $0.16$ magnitudes {\bf SO WHAT}.  In addition, for events where the
redshift of the event was not known, this extra redshift parameter could be
included in the $\chi^2$ calculation to provide a photometric redshift estimate.
\cite{2007A&A...466...11G} finds an RMS dispersion of $\delta~z/(1 + z) =
0.01--0.02$, which is comparable to the redshift that may be inferred from the
spectra alone \citep[see also][]{2010ApJ...717...40K}. Spectral reconstruction
in practice \citep{2010ApJ...719.1759A}.

\begin{figure}[t]
\centerline{\psfig{figure=figures/fall_blowd.eps,width=0.5\textwidth} \hfil
\psfig{figure=figures/rrlyare.eps,width=0.5\textwidth}} \smallskip
\caption[]{\footnotesize Example theoretical spectral--temporal surfaces of
astronomical phenomena.  The {\it left} panel shows the lightcurve of a failed
"fallback" supernovae from \cite{2009ApJ...707..193F}, with total energy of $1.7
\times 10^{51}$ erg, $^{56}$NI yield of $1.0 \times 10^{-13} \msun$ and total
ejecta mass of $3 \msun$.  The {\it right} panel shows a 5--band RR Lyrae
lightcurve template from \cite{2010ApJ...708..717S}.  While these phenomena are
intrinsically quite different, they may be represented in the same
spectral--temporal fashion.  While we anticipate building our surfaces primarily
from experimental data, models like the ones shown here will be used to
fill--in--the blanks when there are insufficient constraints.} \medskip \hrule
\label{fig:sts} \end{figure}


\medskip {\centerline{\ub{\sc Statistical Infrastructure (2p)}}} \smallskip {\bf
Berkeley}

An incoming data stream $f_{i+1}(\lambda, t)$ may be modelled as $$f_{i+1} = x_0
\times ({\bf S_M} + x_1 \times {\bf S_1} + \ldots)$$ with $x_0$ representing the
inferred distance modulus to the event, and $x_1$ representing its location
within the family of events.

Efficient implementation.  Computational issues.  Portable to LSST code (C++ and
Python).

\medskip {\centerline{\ub{\sc Event Classification Service (2p)}}} \smallskip
{\bf Berkeley}

An add-on to what you have running now.  Apply to PTF, Gaia.



\bigskip \centerline{\bf IIIa. Team Qualifications (1p)} \smallskip

\medskip {\centerline{\ub{\sc Dr. Becker}}} \smallskip

PI Becker has an extensive history working on classifying event streams from
within several time--domain projects, including the MACHO project
\citep{2000PhDT.......258B}, the Deep Lens Survey \citep{2004ApJ...611..418B},
and most recently the SDSS--II Supernova Survey
\citep{2008AJ....135..338F,2008AJ....135..348S}.  His most relevant work to this
proposal was in leading the SALT--II cosmology analysis in
\cite{2009ApJS..185...32K}.  His familiarity with this software provided the
inspiration to extend these models to all classes of variability. He has been
aggregating spectral--temporal surfaces for the LSST image simulation effort
\citep{2010SPIE.7738E..53C} for the purposes of added realistic stellar and
cosmological variability to the simulations.  He has been working since 2004 on
the real--time nightly processing pipeline for LSST at the University of
Washington.

\medskip {\centerline{\ub{\sc Dr. Bloom}}} \smallskip

Co--PI Bloom is director of the Berkeley Center for Time--Domain Informatics,
which focuses statisticians, computer scientists, and astronomers on matters of
classification and regression on astronomical time--series.  The primary focus
has been on the real-time and retrospective classification of the Palomar
Transient Factory survey and other public datasets (e.g., Stripe 82). He has
also worked on efficient discovery techniques of quasars through time
variability and fast implementation of Lomb-Scargle periodograms with Bayesian
cross validation. He has worked extensively on gamma--ray burst followup and
characterization, predating the start of the GRB afterglow era in 1997.

\medskip {\centerline{\ub{\sc Dr. Connolly}}} \smallskip

Co-PI Connolly is simulation scientist for the Large Synoptic Survey Telescope,
and lead of the UW Data Management group.  His previous work includes
investigating the dimensionality of large astronomical data sets using PCA and
Locally Linear Embedding, developing and releasing applications for data
intensive cosmology, and for integrating research and education (e.g. Connolly
was the technical lead for the development of Sky in Google Earth).

\bigskip \centerline{\bf IIIb. Previous Support (1.5p)} \smallskip

\medskip {\centerline{\ub{\sc Dr. Becker}}} \smallskip

``The LSST FaST Program : Expanding Participation of Underrepresented Minorities
in LSST'', funded through Specific Program Order 9 (AST-0551161) to the NSF-AURA
(Association of Universities for Research in Astronomy) Cooperative Agreement
AST-0132798.  The project involved simulating tens of millions of RR Lyrae
lightcurves to investigate LSST's ability to recognize the periods and types of
these events, as a function of distance (faintness) and survey duration.  This
team delivered a technical report to the LSST Transients and Variable Stars
working group at the LSST Fall 2009 ``All Hands'' meeting and has submitted a
paper to the Astrophysical Journal on the final results \citep{RRLyrae}.
Importantly, three of the six students funded by this proposal successfully
applied to graduate school, with a fourth expected to apply this next year.

\medskip {\centerline{\ub{\sc Dr. Bloom}}} \smallskip

``Real-time Classification of Massive Time-series Data Streams'' (PI, J. Bloom;
NSF grant No 0941742; amount \$1,573,550; expires 07/12).  Three postdocs, six
graduate students, and several undergraduates have been supported as part of
that project, resulting in more than 10 publications to date.  The
classification framework built as part of this proposal is used in the real-time
pipeline of the Palomar Transient Factory, and has been responsible for the
discovery of more than 15,000 new variable stars and transients.

\medskip {\centerline{\ub{\sc Dr. Connolly}}} \smallskip

"Image Coaddition, Subtraction and Source Detection in the Era of Terabyte Data
Streams" (PI, A. Connolly; NSF grant AST-0709394; amount \$427,933; expires
8/31/2011) is the most closely related grant.  Outcomes from this work include:
the development of non--parametric techniques for the detection of sources
within sequences of astronomical images through the use of image coaddition and
subtraction, and algorithms for measuring the clustering of galaxies using
n-point correlations functions that scale to high-performance parallel
architectures.

\bigskip \centerline{\bf IV. Broader Impacts (1p)} \smallskip \smallskip {\bf
UW}

General event classification.  Understanding the intrinsic dimensionality of
astronomical variability. Photo--z event estimates for things without
spectroscopy.  Note we need a data plan here.

\bigskip \centerline{\bf V. Project Development Plan (1p)} \smallskip

\medskip {\centerline{\ub{\sc Year 1}}} \smallskip

Aggregation and calibration of data

\medskip {\centerline{\ub{\sc Year 2}}} \smallskip

Development of models and statistical infrastructure

\medskip {\centerline{\ub{\sc Year 3}}} \smallskip

Service implementation and verification of models on event streams

