\centerline{\bf I. Introduction} \smallskip

The field of time--domain astrophysics is currently experiencing a renaissance,
driven by the fusion of large datasets, computational infrastructure, and
astro--statistical tools.  Research efforts may be broadly cast into two types
of analyses -- data--mining efforts (discovery) and follow--up of interesting
events (characterization).  The most ambitious of these efforts utilize
real--time discovery tools on survey data streams to drive autonomous follow--up
resources. As these data streams become increasingly more vast and complex, the
statistical tools required to efficiently drive these resources must themselves
increase in complexity.  They must be sensitive to all quadrants of the
Rumsfeldian ``known'' vs.\ ``unknown'' characterization scheme, with the
majority of phenomena falling in the ``known known'' category, and the most
interesting of events falling within the ``unknown unknown''.  To optimally do
this sifting, the complexity of the models must match the complexity of the data
stream to enable a single classification assessment.

In this proposal, we will expand upon the current state--of--the art in event
classification to generate astrophysical variability models with the same
dimensionality as the data being collected. Next generation time--domain data
streams will provide constraints on a given event at irregular times, and in one
of several passbands.  Wide--field time--domain spectroscopic surveys are also
on the horizon.  The event models themselves must therefore be inherently
temporal {\it and} spectral, to federate the ensemble of survey and follow--up
data into a single statistical assessment of event type.  The building of these
models requires adopting a unifying description of astrophysical variability.
{\bf We propose sets of spectral--temporal surfaces, generated for all event
types, as the optimal description of astrophysical variability.}  The process
will incorporate existing data into a single statistical model representing the
mean behavior of each event class, as well as higher--order moments about this
mean that reflect the intrinsic dimensionality of the phenomena.  These models
will be optimal in the sense that they incorporate extant multi--passband
knowledge (photometric {\it and} spectroscopic) into an empirically--derived
event model. They will be optimal in the sense that they represent each class of
phenomena at all times and integrable over any photometric passband, a level of
complexity needed to evaluate the event streams of next--generation surveys such
as the Large Synoptic Survey Telescope
(LSST)\footnote{\url{http://www.lsst.org}}. And they will be optimal in the
sense that they will unify into a common language the statistical description of
astronomical variability from today's heterogeneous standards.  To provide the
broadest possible impact from this effort, we will build a classification
infrastructure incorporating these models into a real--time event broker
available to the astronomical community.


\medskip {\centerline{\ub{\sc A Brief History of Time--Domain Astronomy}}}
\smallskip

The field of time--domain astronomy began in earnest in the 1990s with
wide--field (10 square degree) microlensing surveys such as MACHO
\citep{2000ApJ...542..281A}, OGLE \citep{1994AcA....44..227U}, and EROS
\citep{2003A&A...400..951A}.   Early on, team members recognized that there was
value in a quick concerted response to on--going events: followup after the
event completed would yield none of the second--order effects such as parallax,
resolution of the lensed star's disk, or lens or source binarity.  The
microlensing surveys implemented alert streams that fed (through emails or phone
calls) into follow--up networks such as GMAN \citep{2000PhDT.......258B} and
PLANET \citep{1998ApJ...509..687A}, and which resolved for the first time many
of the exotica expected of complex gravitational lens systems.

Subsequent time--domain efforts turned their focus to precision cosmology
measurements through surveys for Type Ia supernovae \citep{1996AJ....112.2398H},
which most prominently yielded the discovery of the acceleration of the
expansion of the Universe \citep{1998AJ....116.1009R,1999ApJ...517..565P}. These
surveys also required quick ($\sim$few day) analysis of their data to acquire
spectroscopic redshifts of the events. In these cases, the most efficient use of
resources came both by discovery of the event {\it and} with an estimate of the
event type (Ia or otherwise).  Contextual clues were used to assist in event
classification, including proximity to host galaxy and host galaxy type.

Gamma--ray bursts, intense but evanescent flashes from exploding and colliding
stars, are observable across the electromagnetic spectrum but only if
discovered, disseminated, and followed up very rapidly (timescale of seconds).
Intense multi--color imaging coupled with synoptic spectroscopy in the minutes
to hours after events are the state--of--the--art in the field. The most
precious events (scientifically) tend to be those from the lowest and highest
redshifts, providing vistas on star formation, dust obscuration, and potentially
serving as the bridge between the electromagnetic and gravity wave landscapes.
However, at a discovery rate of $\sim$100/yr (primarily from NASA/Swift), the
community cannot adequately follow up all events with the resources available
for target--of--opportunity observations. Knowing instantaneously which GRBs are
likely to have the largest scientific return given substantial telescope
investment is crucial. We (Morgan et al., submitted) have begun using machine
learning tools to try to identify high--redshift events from immediately
available burst data, couching the prediction as a resource maximization
problem.

%morgan: http://qmorgan.org.org/rategrbz/paper.html

With increasing computational resources, surveys are now able to categorize and
release {\it all} types of events found in their data, in real--time. Early
adopters included the Deep Lens Survey \citep{2004ApJ...611..418B} and the Faint
Sky Variability Survey \citep{2003MNRAS.339..427G}.  In these cases, coarse
attempts were made at event classification for all objects displaying
astrometric or photometric variability. Primarily contextual information was
used (e.g. ecliptic latitude, distance from host galaxy) and classifications
were done by humans at the telescope or remotely using web--based visual
classification tools.  More recent time--domain surveys such as the
Palomar--Quest survey \citep{2008AN....329..263D} and Catalina Real-Time
Transient Survey \citep{2011arXiv1102.5004D} have begun to adopt modern
statistical techniques to classify events.  This includes morphological
classification of the pixel--level flux that triggered the event
\citep{2008AIPC.1082..252D}, as well as contextual information regarding the
location of the variability \citep{2010ASPC..434..115M}.  We note that colors of
the quiescent objects are used in these classification schemes, but {\it not}
the time--varying color of the variable flux.

%Kepler classification? \citep{2010ApJ...713L.204B}

In the NSF--sponsored Palomar Transient Factory, we have begun to use machine
learning techniques to discover new events through image differencing,
identifying roughly 1000 variable stars and transients out of 1.5 million
candidates per night. These events are, in real--time, classified with a
maximum--likelihood classifier that makes use of contextual information (such as
color of the nearest galaxy) and temporal metrics (such as difference in
magnitude between the event and the quiescent counterpart brightness).  We have
shown \citep{2011arXiv1106.5491B} the ability to distinguish between transients
and variable stars with a 3.8\% overall error rate (with 1.7\% errors for
imaging within the Sloan Digital Sky Survey footprint).  At $>$96\%
classification efficiency, the samples achieve 90\% purity.  Determining just
what sort of transient is found (e.g., supernova, nova, QSO event) has proven
more challenging.  Only with retrospective analysis, once sufficient data on a
light curve have been obtained, do the classification errors begin to drop
significantly \citep{2011ApJ...733...10R} (see \citealt{2011arXiv1104.3142B} for
a review).

In all cases mentioned above, the volume of data being collected made the
surveys sensitive to new types of phenomena, with a rapid response component
that enabled immediate and early study.  It is also the case that many of these
efforts designed their event filters based upon the particular survey design
they were being applied to.  To continue this trend into the future, where
surveys such as Gaia\footnote{\url{http://gaia.esa.int}} and LSST will report on
{\it all} classes of variability, a class of models needs to be generated that
match the ambitious designs of the surveys (massive data rates, and sampling
that is sparse in time and inhomogeneous in wavelength).

\medskip {\centerline{\ub{\sc Strides Towards Spectral--Temporal Event
Classification}}} \smallskip

While teams such as the Harvard Time Series
Center\footnote{\url{http://timemachine.iic.harvard.edu/}} and the Berkeley
Transients Classification
Pipeline\footnote{\url{http://dotastro.org/about/tcp.php}} have made substantial
advances in automated event classification, their methods remain inherently
one--dimensional. That is, their models of event behavior do not yet take into
account color evolution during the events.

One field where there has been substantial progress in spectral--temporal
modeling is in the field of supernova studies. The pioneering work of
\cite{2002PASP..114..803N} integrated a large, inhomogeneous sample of
spectroscopy of Ia supernovae into a single averaged representation of the Ia
phenomena. This integrated model represented the behavior of a typical
``Branch--normal'' Type Ia supernova, and defined the spectrum of a typical
event at all integer wavelengths between 1000~\AA~and 25000~\AA, 1 per day for
90 days after explosion.  Nugent subsequently expanded his template set to
include separate models for intrinsically bright and faint Ia, supernovae Type
Ib/c, and Types II P/L/n. Such templates played a crucial role in real--time
event classification for the SDSS-II Supernova Survey
\citep{2008AJ....135..338F}, and enabled a $90\%$ targeting efficiency for Type
Ia supernova \citep{2008AJ....135..348S}. However, since they only existed for
supernova--class events, all events that did not conform to the spectral and
temporal behavior of these models were subsequently ignored.

This notion of spectral--temporal event representations was expanded upon  by
\cite{2007A&A...466...11G} in the generation of their SALT--II supernova model
(see also \cite{2007ApJ...663.1187H}).  This model incorporated spectroscopic
{\it and} photometric data from low-- and high--redshift supernovae to yield a
set of lightcurve templates describing the behavior of {\it all} Type Ia
supernovae. The spectroscopic data constrain the templates sparsely in time
(being acquired infrequently compared to photometry) but finely in wavelength,
while the photometric data constrain the templates more densely in time, but
only coarsely in wavelength.  The photometric data were also able to be
calibrated, in an absolute sense, to a much higher accuracy than the
spectroscopic data, and were used to bootstrap the overall calibration of the
model.  These data in combination yielded a highly constrained two--dimensional
surface in time and wavelength that describes the temporal evolution of the
rest-frame spectral energy distribution (SED) for SNe Ia. Importantly, the
SALT--II model yields not just the average behavior of the Type Ia sample,
equivalent to the \cite{2002PASP..114..803N} data, but also includes a first
``principal component'' about the mean, which models the diversity
(intrinsically bright to intrinsically faint) of Ia. This single representation
-- mean surface plus a first moment -- contains the behaviors and correlations
seen in the Ia population such as the $\Delta m_{15}$ brightness--decline
\citep{1993ApJ...413L.105P} or brightness--stretch \citep{1998AJ....116.1009R}
relations.  The need for only a single principal component reflects the fact
that Ia supernova are intrinsically a one--parameter family; the SALT--II model
captures this in a spectral--temporal representation.

Figure~\ref{fig:salt2} displays the mean surface ({\it left}) and first
principal component surface ({\it right}) from the SALT--II model.  The behavior
for a given Ia lightcurve may be generated by an overall scaling of the mean
surface (representing distance modulus) with a contribution from the secondary
surface representing its place within the 1--parameter family of Ia supernovae.
Importantly, an incoming spectral--temporal data stream can be compared to these
surfaces, and a statistical assessment made whether or not it is behaving in
accordance with this model.  This likelihood may be generated using the
surfaces, the photometric uncertainty on the incoming data, and (optionally)
priors on the amplitude of the second surface as indicated by the best fit. This
is the process we envision resulting from this proposal : {\bf an incoming event
stream will be evaluated against an ensemble of these surfaces to yield a
statistical likelihood that the event is of each class.}


\begin{figure}[t]
\centerline{\psfig{figure=figures/surface0.eps,width=0.5\textwidth} \hfil
\psfig{figure=figures/surface1.eps,width=0.5\textwidth}} \smallskip
\caption[]{\footnotesize Spectral--temporal surfaces of Type Ia supernovae from
the SALT--II model of \cite{2007A&A...466...11G}.  Rest--frame SED surfaces are
generated from the aggregation of photometric and spectroscopic data on several
hundred vetted Ia supernova.  The leftmost figure represents the mean behavior
of the sample, defined every 10~\AA~between 2000~\AA~and 9200~\AA, and in daily
intervals from -20 days from B--band peak brightness to +50 days.  The right
figure displays the first moment of these data about the mean distribution,
derived using principal component analysis.  The diversity of Ia lightcurves
(across all passbands and at all times) can be represented by a scaled addition
of this secondary surface to the primary surface.  {\bf We propose here to make
similar models for {\it all} classes of astronomical variability.}} \medskip
\hrule \label{fig:salt2} \end{figure}

\bigskip \centerline{\bf II. Proposed Work} \smallskip

We propose to make spectral--temporal (ST) surfaces for {\it all} major classes
of astronomical variability.  The individual components of this effort include:

\begin{itemize}

\item the aggregation of photometric and photometrically--calibrated
spectroscopic data from a diverse range of input sources;

\item the generation of a spectral--temporal variability model from all data on
a given class;

\item an investigation into the dimensionality of each class, to determine how
many principal components to include in the model;

\item a statistical framework to generate likelihoods that an incoming event
stream is consistent with each variability model;

\item and an on--line classification service that includes our
spectral--temporal models as part of an overall (single--epoch and multi--epoch)
event assessment.

\end{itemize}

These surfaces will serve as common currency in future event classification
efforts, alongside extant efforts such as: Artificial Neural Network
\citep{2008AN....329..263D}, Support Vector Machine \citep{2007ApJ...665.1246B},
and human--vetted \citep{2011arXiv1106.5491B} classifiers of pixel--level
artifacts; contextual understanding of new single--epoch events using Markov
Logic Networks \citep{2011arXiv1110.4655D} and feature--based classifiers
\citep{2011arXiv1106.5491B}; and shape--based Random Forest
\citep{2011ApJ...733...10R} classifiers of lightcurves in a single passband. Our
effort will add an extra dimensionality to these state--of--the--art
classification efforts, allowing the incorporation of information at multiple
epochs and wavelengths into the overall classification scheme.  Below we
describe in detail each step in this process.


\medskip {\centerline{\ub{\sc Aggregation of Data}}} \smallskip

The first step in this process is to identify and aggregate the diversity of
time--domain data.  We will use as an initial resource photometry that has been
collected by the Berkeley Time Domain Center\footnote{\url{
http://dotastro.org/lightcurves/classes.php}}.  This includes data from Galactic
and extra--Galactic variable star catalogs such as \cite{2008yCat.2285....0B},
\cite{2001AJ....121..870M}, etc., subdivided into 137 subclasses.

\begin{figure}[t]
\centerline{\psfig{figure=figures/vartree.eps,width=1.0\textwidth}} \smallskip
\caption[]{\footnotesize A tree--based representation of astronomical
variability, from \cite{2008JPhCS.118a2010E}.  We will build spectral--temporal
models for these phenomena, as well as their subclasses, from vetted data
sources.} \medskip \hrule \label{fig:var} \end{figure}

Additional input data will include, but not be limited to: re--calibrated
photometry from SDSS Stripe 82 on variable stars \citep{2010ApJ...708..717S} and
supernovae \citep{2008AJ....136.2306H,2011ApJ...738..162S}, including
per--camera filter profiles of \cite{2007AJ....134..973I}; variable stars from
the OGLE and Hipparcos surveys \citep{2007A&A...475.1159D}; photometric and
spectroscopic data from the Carnegie Supernova Project
\citep[e.g.][]{2010AJ....139..519C} and Supernova Legacy Survey
\citep[e.g.][]{2011yCat..74101262W}; the Suspect Supernova Spectrum
Archive\footnote{\url{http://suspect.nhn.ou.edu/~suspect/}}; low--redshift
supernova data from \cite{2006AJ....131..527J}; and classes of novae including
recently discovered faint--fast \citep{2011ApJ...735...94K} and luminous red
novae \citep{2011ApJ...730..134K}.  The Co--Is are also involved in proposals
for time--domain spectroscopic surveys, with source selection based upon
variability catalogs from e.g. SDSS Stripe 82 and
Pan--STARRS\footnote{\url{http://pan-starrs.ifa.hawaii.edu/public/}}; while
these would be optimal spectral inputs to our models, we do not require them for
this effort to be a success.  Although the above list is not exhaustive, it does
give an idea of the scope of the effort we intend to undertake --
Figure~\ref{fig:var} provides a representation of this variable menagerie. In
addition, many of the ``known unknown'' classes of variability have theoretical
models we must incorporate into the spectral--temporal paradigm.  This includes
a large variety of super-- and sub--luminous transient events
\citep[e.g.][]{2010ApJ...715..767S,2009ApJ...707..193F}.  It will be especially
important to recognize an instance of these classes of events from the more
common transients, since early observations will be key to testing
currently--unconstrained theoretical models.  Example model lightcurves are
shown in Figure~\ref{fig:sts}.

\begin{figure}[t]
\centerline{\psfig{figure=figures/fall_blowd.eps,width=0.5\textwidth} \hfil
\psfig{figure=figures/rrlyare.eps,width=0.5\textwidth}} \smallskip
\caption[]{\footnotesize Example model spectral--temporal surfaces of
astronomical phenomena.  The {\it left} panel shows the lightcurve of a failed
``fallback'' supernovae from \cite{2009ApJ...707..193F}, with total energy of
$1.7 \times 10^{51}$ erg, $^{56}$NI yield of $1.0 \times 10^{-13} \msun$ and
total ejecta mass of $3 \msun$.  The {\it right} panel shows a 5--band RR Lyrae
lightcurve template from \cite{2010ApJ...708..717S}.  While these phenomena are
intrinsically quite different, they may be represented in the same
spectral--temporal fashion.  While we anticipate building our surfaces primarily
from experimental data, theoretical models will be used to ``fill in the
blanks'' when there are insufficient data.} \medskip \hrule \label{fig:sts}
\end{figure}

As we will be combining experimental data taken on different instruments, in
different filters, and at different epochs, we must enforce a strict set of
standards to ensure we can merge these data into a single ST model. In practice,
this will limit our sources of data to archives where calibration metadata is
part of the data curation. All incoming photometric data must be accompanied by
a profile corresponding to the transmission of the filter, as well as a time
stamp corresponding to the epoch of observation. We will convert each filter
profile into an overall system throughput representing the dimensionless
probability that a photon of wavelength $\lambda$ reaches the detector.  This
filter profile will be used to weight each data point's contribution to the ST
surface.  We will also use each filter profile to convert (if necessary) the
flux into the AB magnitude system. We will convert each  time stamp to
correspond to the mid--point of observation, in Barycentric Julian Date in the
Barycentric Dynamical Time standard \citep{2010PASP..122..935E}. All incoming
spectroscopic data will have the additional requirement that it has been
spectro--photometrically calibrated (or is able to be calibrated using
near--concurrent photometric data).  In practice, this is very difficult to
accomplish, and will limit the amount of spectral data we are able to use (i.e.
``one--off'' spectra of events to determine redshift are not necessarily taken
with the slit aligned with the parallactic angle, a requirement for
spectrophotometric calibration; \citealt{1982PASP...94..715F}).

An additional requirement for objects at cosmological distances is that any
photometric data must be accompanied by a redshift to translate the
observer--frame filter profile into a rest--frame window. This has the
disadvantage that K--correction of the photometry \citep{2002astro.ph.10394H}
requires an instantaneous spectrum of the event, which leads to a
chicken--and--egg problem since that exact spectrum is expected to result from
the ST modeling.  This will require an iterative boot--strapping of the
cosmological K--corrections with an initial ST surface.  However, the redshift
effect will allow us to build ST surfaces that are well--calibrated in the
rest--frame u--band, where ground--based calibration is difficult, by using
photometry of objects at moderate redshift.

\medskip {\centerline{\ub{\sc Generation of Spectral--Temporal Surfaces}}}
\smallskip

We proceed using the following underlying model for astrophysical variability:
the flux from an astronomical object may be represented as $F_\nu(\lambda, t)$
with $F_\nu$ the specific flux of an object above the atmosphere, in Janskys (1
Jy = $10^{-23}$ erg cm$^{-2}$ s$^{-1}$ Hz$^{-1}$), as a function of wavelength
and time.  Our spectral--temporal models will be defined over the grid of
$\lambda$ and $t$, with the bin sizes dependent on the amount of input data
available.  These models will be defined in the event rest--frame, and (to the
best of our ability) above the Earths's atmosphere, meaning the input data will
need to be calibrated in an absolute sense. Data on a single event will only
coarsely sample this surface.  However, the phase space will be more densely
sampled by an ensemble of data of different events of the same class.

Between emission and measurement, $F_\nu$ is attenuated by a (typically unknown)
atmospheric transmission function $T^{atm}(\lambda, t)$, which is the
probability that a photon of wavelength $\lambda$ successfully propagates
through the atmosphere, and a (typically measured, and occasionally
well--measured) system transmission probability per unit photon
$T^{sys}_{filt}(\lambda, t)$.  This latter term should include the wavelength
dependence of the mirror reflectivity, lens and filter transmission, and
detection sensitivity.  The product of these two defines the overall
transmission profile of a given observation in a given filter $T_{filt}(\lambda,
t)$.  We note that some surveys such as SDSS define their filter profiles with
the atmospheric component rolled in \citep{2007AJ....134..973I}.  We will assume
that calibrated photometry has had atmospheric and system transmission accounted
for in the zeropointing, to the best of the surveys' ability.  The total counts
transmitted through the optical system are then: $$F^{obs}_{filt} (t) = \int
F_\nu(\lambda, t) ~~ T_{filt} (\lambda, t) ~~ \lambda^{-1} d\lambda$$ with
$T_{filt}(\lambda, t)$  the overall transmission probability per unit photon
\footnote{The term $\lambda^{-1}$ comes from the conversion of energy per unit
frequency into the number of photons per unit wavelength}. To place a flux
measurement on the AB magnitude system, flux is normalized by $F_{AB} = 3631$
Jy.

From the available data on each input training event, we will create a sparse
surface labelled $f(\lambda, t)$ that constrains the overall model.  Spectral
data will be rebinned to the wavelength resolution of our model, and photometric
flux will added to the model across wavelengths with a weighting of
$T_{filt}(\lambda, t)~\lambda^{-1}$.  To generate the ST model from these data,
we proceed with the standard assumption that each input dataset $f_0(\lambda,
t), f_1(\lambda, t)~\ldots~f_i(\lambda, t)$ may be recovered from a linear
combination of (as--yet unknown) basis surfaces $$f_i(\lambda, t) = \sum_{j=0}
x_{ij} \times {\bf S_j}(\lambda, t)$$ where ${\bf S_j}(\lambda, t)$ are said
basis surfaces and $x_{ij}$ are their weightings (which will be different for
each given event).  By choosing the right basis, we may approximate each event
with a finite number of surfaces ${\bf S_j}(\lambda, t)$. Principal Component
Analysis (PCA) is an optimal way to select this basis from an ensemble of input
data \cite[e.g.][for astrophysical application to galaxy
spectra]{1995AJ....110.1071C}, and ``sparse'' or ``gappy'' PCA is a particular
implementation to be used when the inputs sample only a small fraction of the
model \citep[e.g.][]{zouht04}, such as here.  As is standard when using PCA
decomposition, we will first create a ``mean'' surface ${\bf S_M}(\lambda, t)$
and find the principal variations about this mean, yielding the model:
$$f(\lambda, t) = x_0 \times \left( {\bf S_M}(\lambda, t) + \sum_{j=1} x_j
\times {\bf S_j}(\lambda, t) \right)$$ where $x_0$ represents an overall
brightness scaling (i.e. distance modulus), and $x_j$ represents the location of
the phenomena within the family of events.

\begin{center} {\bf We emphasize that these basis surfaces (${\bf S_M}(\lambda,
t)$,${\bf S_1}(\lambda, t)$,${\bf S_2}(\lambda, t)$ $\ldots$ ${\bf S_j}(\lambda,
t)$) are the fundamental products of this proposal.} \end{center}

Algorithmically, the process to generate these surfaces is as follows:
accumulate data on each instance of the event type to be modelled; for each
instance, create a sparse data surface $f(\lambda, t)$; from the ensemble of
$f_i(\lambda, t)$, find the mean ${\bf S_M}(\lambda, t)$; subtract the mean
${\bf S_M}(\lambda, t)$ from all $f_i(\lambda, t)$; run a Karhunen--Loeve
decomposition \citep{Karhunen:47,Loeve:48} on the mean--subtracted $f_i(\lambda,
t)$, yielding principal surfaces ${\bf S_j}(\lambda, t)$.

\smallskip {\bf Intrinsic Dimensionality of each Basis:} The Principal Component
Analysis is fundamentally an eigenvector analysis, with eigenvalues that
represent the variance in the population represented by each surface. Surfaces
that represent effects that dominate the behavior of the population (e.g. the
brightness--decline relationship for Supernova Ia) will have large eigenvalues,
with secondary effects less important and therefore having smaller corresponding
eigenvalues in the PCA.  One can thus derive the intrinsic dimensionality of the
population from the spectrum of eigenvalues. Correspondingly, one can well
approximate any event within this population by using the first few bases
(sorted by eigenvalue).  We will use such a truncated expansion when creating
our basis models; only surfaces representing the majority of the variance
($90\%$ to $99\%$, depending on the amount of input data) will be retained in
the event modeling process described above.

\smallskip {\bf Uncertainties on the Surfaces:} We will additionally identify
excess variance in the models that is not represented by the surfaces, e.g.
portions of large model uncertainty that we will want to de--weight when
fitting. This excess variance will be modelled using a jackknife resampling
procedure: for each input event, we will remove it from the training sample and
re--create the model.  The variance of the models and  residuals across all
jackknife resamplings will be used as an empirical estimate of the model
uncertainty.  For theoretical surfaces, we may generate uncertainties by ranging
over plausible combinations of input parameters, although we will have to be
careful to not make the surfaces too exclusive.

\smallskip {\bf Variants to this Procedure:} We recognize that for certain
phenomena, this basic derivation of the surfaces is not sufficient to completely
describe the events. For example, objects with periodic variability require
folding at the correct period before fitting to these surfaces.  We must
therefore construct periodograms for these lightcurves, optimally using the
ensemble of event data to yield a single period assessment.  This will require
development of multi--band period assessment tools.  Two options we will pursue
are: the generation of periodograms for each passband of information, the
conversion of these periodograms into likelihood functions using a false--alarm
probability analysis \citep[e.g.][and references therein]{2009A&A...496..577Z},
and using the product of these likelihood functions in a final period
assessment; or to use priors on color--evolution derived from our ST surfaces to
wrap all data into a single periodogram analysis.  Other classes of events such
as supernovae undergo substantial extinction from their host galaxy, which will
create a $\lambda$--dependent non--intrinsic reddening of the underlying ST
surface. This will require an assessment of the reddening of each input vector,
and an additional fit parameter on each incoming lightcurve representing the
degree of extinction \cite[e.g. Equation 1,][]{2007A&A...466...11G}.  Finally,
the evolution of objects at cosmological distances will behave like a stretched
spectral--temporal surface.  When fitting such classes of events, we will
compare the observed data to ST models with an extra degree of freedom in the
redshift $z$, e.g. $f(\lambda, t)$ vs. ${\bf S_M}(\lambda/(1+z), t/(1+z))$.

\smallskip {\bf Previous Work:} We note that this procedure follows very closely
the SALT--II model of \cite{2007A&A...466...11G}.  In SALT--II, the degree to
which the model captured the behavior of the training data is reflected in the
RMS about the Hubble diagram of $0.16$ magnitudes \citep{2007A&A...466...11G}.
This compares favorably to an RMS $\sim 0.2$ magnitudes for competing methods
\citep{2009ApJS..185...32K}, indicating the model is able to advance the
state--of--the--art even in the competitive field of supernova cosmology. In
addition, for events where the redshift of the event was not known, this extra
redshift parameter was included as a degree of freedom in the $\chi^2$
minimization to provide a photometric redshift estimate.
\cite{2007A&A...466...11G} finds an RMS dispersion of $\Delta z/(1 + z) =
0.01-0.02$, which is comparable to the results that may be inferred from the
spectra alone \citep[see also][]{2010ApJ...717...40K}.  A slightly different
implementation of this process is described in \cite{2010ApJ...719.1759A}, where
they are able to reconstruct stellar spectra by comparing 3 broad--band
magnitudes to a set of spectral basis functions derived using PCA
\citep{2010AJ....139.1261M}.  Obviously, PCA is a powerful diagnostic when
applied to spectral data; we intend to expand this utility into the
spectral--temporal regime.

\medskip {\centerline{\ub{\sc Statistical Infrastructure}}} \smallskip

We expect to construct the set of  ${\cal S}_k \equiv \{({\bf S_M},{\bf
S_j})\}_k$ for $k$ classes of variables and transients. While the actual number
of classes will ultimately be subject to availability and quality of the input
data sets, we expect $k \approx 100$ and may be much more if we include a
variety of theoretical curves. The fundamental question we wish to answer is:
{\it given a new light curve, $f_{i+1}(\lambda, t)$, what is the probability
that it arises from class $k$?}  A related question we wish to answer is: {\it
what is the probability that it does not arise from any known class?}

A straight forward approach is to represent a notion of distance of the
lightcurve from each of the $k$ templates as ${\cal D}_k(f_{i+1})$, where the
normalization and start time $t_0$ (or phase offset $\phi_0$, or extinction, or
redshift $z$) are varied so as to minimize the $\chi^2$ for each ${\cal S}_k$.
The most likely class is then that which minimizes ${\cal D}$. In the limit with
well--sampled and multicolor light curves, this approach should be sufficient
for class {\it selection}. However, for probabilistic lightcurve classification,
especially in the few--data limit, a more complex approach is required: in
particular, there must be some notion of a minimal distance subject to the
likelihood of having observed the particular dataset.  For example, each
spectral--temporal training set yields the expected range of contributions from
the secondary (tertiary, etc.) surfaces $x_j$, which may be used as priors when
doing absolute classification. We will explore such a Bayesian approach to
distance minimization.

The lightcurve classification problem raises a very interesting statistical
question: how can we account for the taxonometric (hierarchical, tree--like,
Figure~\ref{fig:var}) structure of the landscape of astronomical events? A
increasingly more detailed set of questions may be asked about such an event: is
it variable or non--variable? Is it a pulsating variable or an eclipsing binary?
Is it a Cepheid or an RR Lyrae? What kind of RR Lyrae is it?  With limited data
it may be possible to make highly probable classifications at the high levels of
the taxonomy but not at the lowest levels.

The General Catalog of Variable Stars (GCVS; \citealt{ksf+96}) has developed a
classification tree based upon an admixture of observed properties (e.g.,
``slow'' and ``red'') and inferred physical characteristics (e.g.,
``eclipsing'').  While it is not crucial for our purposes that a taxonomy be
based on intrinsic physical properties alone, the lack of a coherent
taxonometric structure for variable sources presents a challenge when one of the
goals is to make a probabilistic statements about the nature of an unknown
event. {\bf We propose to explore possible taxonomies with particular attention
to their computational consequences}.  The existing taxonometric structure is
certainly relevant to training a classifier based upon ${\cal S}_k$, since not
all lightcurves in the training database will be confidently identified at the
leaf level, but may contain valuable information about the relationships of
features to higher level vertices in the taxonomy.

Making robust probabilistic statements based on a graphical structure has
received some attention in the statistics literature
\citep{DBLP:conf/icml/KollerS97,DBLP:conf/icdm/SunL01,DBLP:conf/icml/DekelKS04,
CesaBianchi06,  bcmt-lmmsc-04}, with much of the interest spurred by the
problems of identifying text (e.g., \citealt{593971}). We propose to explore two
general ways of attacking this problem. One approach is to incorporate a fixed
(e.g. GCVS) taxonomy by changing our measures of loss to reflect taxonomic
information (through either simple tree--based distances or more involved
Laplacian--based ideas), going beyond the standard ``0--1'' loss (where the cost
of misclassifying an RR Lyrae as a Cepheid is the same as failing to distinguish
between two types of RR Lyrae).  Another approach addresses the taxonomy by
doing a ``multi--scale'' or ``multi--resolution'' classification.  Much of the
exploration here will be in finding the optimal metrics on the tree (e.g.
\citealt{ChungSpectralGraphTheoryBook97}).

Another approach to incorporating taxonomy into classification, which we propose
to explore, would make use of recent work on statistical phylogenetics, where
the taxonomy is provided by a phylogenetic tree\citep{elithesis}.  The idea is
to use the taxonomy to induce a meaningful inner--product between observations.
This is done by, for instance, using the Laplacian of the graph and its spectral
decomposition to find a taxonomically meaningful way of representing functions
on the graph, which for us are really vectors of observations on the graph. When
using PCA one can then incorporate graph information into classical multivariate
statistical techniques \citep{PurdomAnalyzingDataWithGraphs08}.  A natural
question is to investigate whether a similar idea can be used in the context of
``max--margin'' classifiers. We note that ideas from structured classification
work \citep{bcmt-lmmsc-04} will be helpful in this context. In this line of
inquiry, we might also ask if one can derive linear combinations of features
that are taxonomically meaningful and use those in growing decision trees.

Another possible approach that takes into account the tree structure of the
problem would consist of pruning the tree ${\cal T}_0$ of ST surfaces at
different levels and constructing recursively classifiers for each level.  For
example, we could first construct a random forest that would classify the
observations into the first level of categorizations of the tree ${\cal T}_0$.
Call $\{{\cal T}_1^{(i)}\}_{i=1}^k$ the corresponding subtrees. We could then
fit a random forest in the same fashion to classify the observations assigned to
each of these new trees into their first level of categorization. A procedure
like this one clearly takes into account the taxonomy information (cf.\
\citealt{CesaBianchi06}).  We note that this sequence of classifiers could be
constructed with a sequence of distance--based loss functions described above
which assign edge lengths equal to zero for lower levels of the tree.

The notion of a classifier being ``calibrated'' is a relevant line of inquiry.
This has long been a concern in probabilistic weather forecasting -- does it
rain on 30\% of the days for which the forecast probability of rain was 30\%?
The ``probabilities'' a classifier produces may not correspond to the actual
relative frequencies in the training set on which it was based
\citep{niculescu05:obtaining}.  In our context it would be desirable that the
proportion of objects labeled as pulsating variables is a good estimate of the
actual proportion of such variables in the data set.  For example, we seek  to
meaningfully combine  the reported probabilities of the various types of
pulsating variables to obtain the marginal probability that an object is of this
type.  We plan to investigate this issue of calibration within a taxonomic
structure.

The nexus of statistical methods, modern computational resources, and massive
data sets is leading to revised notions of statistical efficiency that take into
account computational constraints \citep{MeinBickRice2008}.  Convex surrogates
for the 0--1 loss function as mentioned above are but one of a growing number of
examples. Mindful of the importance of computational efficiency in the LSST era,
{\it we expect that the methods we develop to meet the unique challenges posed
by lightcurve classification will contribute to this evolving literature.}

Computational constraints are less important during the learning/training phase
than at other stages.  A decision tree would clearly be preferable to a Bayesian
method that required expensive Markov Chain Monte Carlo (MCMC) runs for every
object. Constraining a classification procedure by a computational cost would
amount to adding a Lagrange multiplier times the computational cost to the
statistical loss function.  The key is that the resulting optimization problem
be computationally tractable.   \citet{MeinBickRice2008} show that by using
dynamic programming,  tree searches from the top node downward  could produce
enormous gains in computation with very little loss of statistical efficiency.

An especially intriguing and challenging aspect of real--time lightcurve
classification is that the ${\cal D}_k$ of each object \emph{changes} with every
new data time--series data point. Our family of surfaces should in fact suggest
which types of observations, in wavelength and time, would be optimal in driving
the evolution of ${\cal D}_k$, thus optimizing the allocation of follow--up
resources.  Here the taxonomic approach can be used to advantage. For example,
some strategies arise naturally for an approach that composes classification
procedures at every vertex of the tree. For a given object it may not be too
important to classify more finely in the taxonomy at every new observation.  On
the other hand, if there is some non--negligible probability of an event
belonging to a rare class or a never--observed phenomena, it would be crucial to
update frequently.  Thus, updating policies can vary among the vertices and the
updating choices will depend dynamically on the availability of computational
resources. We propose to explore this technique both at the theoretical level
and by example.

\medskip {\centerline{\ub{\sc Event Classification Service}}} \smallskip

Members of our collaboration are already running a discovery and classification
engine for the Palomar Transient Factory and will soon be adding the Dark Energy
Survey and La Silla Supernova Search.  This service, now private, is highly
trained to PTF cadences and peculiarities, making use of light curve and context
information when available. The ${\cal D}_k$ metrics on each source will serve
as additional features for the classification engine.

Both DES and La Silla are expected to run during much of the lifetime of the
proposed work, allowing us to expand the classification effort with this
additional input. In addition, as new surveys start producing public streams
(e.g., PTF2 and Gaia) we intend to add a classification engine for those sources
as well.  For all non--proprietary data streams we propose to build a public
classification service, allowing subscribers to receive (via push and pull
mechanisms) real--time classification statements about each source that match
certain criteria. This web service will be operated out of UC Berkeley and make
use of best practices for web front ends, database architecture (e.g.,
postgresql, Hadoop), push mechanisms (e.g., Jabber), and astronomical event
packaging (e.g., VOEvent and other VO tools).

%\smallskip % hack to get stuff to line up

\bigskip \centerline{\bf III. Broader Impacts} \smallskip

This proposal will build and strengthen collaboration between current producers
and consumers of real--time event data (UCB with PTF) and future producers of
same (UW with LSST).  It will bring needed statistical rigor to the
multidimensional description and interpretation of astrophysical phenomena; this
type of interdisciplinary effort typically pays dividends by fostering advances
in both fields.  In statistics this will occur in the application of sparse PCA
on a two--dimensional surface of information, and in the innovation that it will
require to operate in a computationally efficient and robust manner.  In
astronomy, it will result in the understanding of the intrinsic dimensionality
of all classes of astronomical variability, how the classes may be interrelated,
and what sort of observations are optimal for nailing down the class of a given
event.  These models will also enable photometric redshift estimates for the
event types \citep[e.g.][]{2010ApJ...717...40K}, which may be used in
conjunction with photometric redshifts of any host galaxy
\citep[e.g.][]{1962IAUS...15..390B} to provide a more certain estimate of event
redshift without spectroscopic observations, which are expensive to obtain.

Our participation in current and future time--domain surveys will also ensure
that these models are not generated in a vacuum, and that they will be developed
with practicality in mind.  We plan to advertise this project at multiple
conferences per year, which will allow for collaboration with other producers
and consumers of time--domain data we may have not included here. Importantly,
our planned inclusion of these models into extant classification services will
ensure that they achieve broad application and relevance in a burgeoning field.


\bigskip \centerline{\bf IV. Team Qualifications and Previous Support}
\smallskip

\medskip {\centerline{\ub{\sc Dr. Becker}}} \smallskip

University of Washington PI Becker has an extensive history working on classifying event streams from
within several time--domain projects, including the MACHO project
\citep{2000PhDT.......258B}, the Deep Lens Survey \citep{2004ApJ...611..418B},
and most recently the SDSS--II Supernova Survey
\citep{2008AJ....135..338F,2008AJ....135..348S}.  His most relevant work to this
proposal was in leading the SALT--II cosmology analysis in
\cite{2009ApJS..185...32K}.  His familiarity with the SALT--II software provided
the inspiration to extend these models to all classes of variability, as
proposed here. He has been aggregating spectral--temporal models for the LSST
image simulation effort \citep{2010SPIE.7738E..53C} for the purposes of adding
realistic stellar and cosmological variability to the simulations.  He has been
working since 2004 on the real--time nightly processing pipeline for LSST at the
University of Washington.

{\bf Previous Support:} ``The LSST FaST Program : Expanding Participation of
Underrepresented Minorities in LSST'', funded through Specific Program Order 9
(AST-0551161) to the NSF-AURA (Association of Universities for Research in
Astronomy) Cooperative Agreement AST-0132798.  The project involved simulating
tens of millions of RR Lyrae lightcurves to investigate LSST's ability to
recognize the periods and types of these events, as a function of distance
(faintness) and survey duration.  This team delivered a technical report to the
LSST Transients and Variable Stars working group at the LSST Fall 2009 ``All
Hands'' meeting and has submitted a paper to the Astrophysical Journal on the
final results \citep{RRLyrae}. Importantly, three of the six students funded by
this proposal successfully applied to graduate school, with a fourth expected to
apply this next year.

\medskip {\centerline{\ub{\sc Dr. Bloom}}} \smallskip

UC Berkeley PI Bloom is director of the Berkeley Center for Time--Domain Informatics,
which focuses statisticians, computer scientists, and astronomers on matters of
classification and regression on astronomical time--series.  The primary focus
has been on the real--time and retrospective classification of the Palomar
Transient Factory survey and other public datasets (e.g., Stripe 82). He has
also worked on efficient discovery techniques of quasars through time
variability and fast implementation of Lomb--Scargle periodograms with Bayesian
cross validation. He has worked extensively on gamma--ray burst followup and
characterization. He is co--chair of the LSST science working group on
transients and variable stars. VOEvent, the IVOA standard for astronomical event
publication, was his brainchild.

{\bf Previous Support:} ``Real-time Classification of Massive Time-series Data
Streams'' (PI, J. Bloom; NSF grant No 0941742; amount \$1,573,550; expires
07/12).  Three postdocs, six graduate students, and several undergraduates have
been supported as part of that project, resulting in more than 10 publications
to date.  The classification framework built as part of this proposal is used in
the real-time pipeline of the Palomar Transient Factory, and has been
responsible for the discovery of more than 15,000 new variable stars and
transients.

\medskip {\centerline{\ub{\sc Dr. Connolly}}} \smallskip

Co-PI Connolly is simulation scientist for the Large Synoptic Survey Telescope,
and lead of the UW Data Management group.  His previous work includes
investigating the dimensionality of large astronomical data sets using PCA and
Locally Linear Embedding, developing and releasing applications for data
intensive cosmology, and for integrating research and education (e.g. Connolly
was the technical lead for the development of Sky in Google Earth).

{\bf Previous Support:} ``Image Coaddition, Subtraction and Source Detection in
the Era of Terabyte Data Streams'' (PI, A. Connolly; NSF grant AST-0709394;
amount \$427,933; expires 8/31/2011) is the most closely related grant. Outcomes
from this work include: the development of non--parametric techniques for the
detection of sources within sequences of astronomical images through the use of
image coaddition and subtraction, and algorithms for measuring the clustering of
galaxies using n-point correlations functions that scale to high-performance
parallel architectures.

\bigskip \centerline{\bf V. Project Management and Development Plan} \smallskip

PI Becker will serve as the lead for supervision of a research associate (at the
post--doctoral level) at UW.  This postdoc ideally will have expertise in the
calibration and management of survey--level volumes of astrophysical data.
PI Bloom will serve as the lead for supervision of a second research
associate (at the post--doctoral level) at UCB.  This postdoc ideally will have
expertise in the application of statistical tools in the field of astronomy.  We
require RAs at the postdoctoral level because the technical challenges posed by
this problem (data calibration, computational implementation) require
individuals with immediate expertise, and who have fought through such issues
before at (at least) the graduate student level. PI Becker will serve as the
technical lead for the project, and will contribute at the level of 25\% of his
time.

The difficulties in having a geographically distributed team require a
commensurate level of interaction to maintain the group focus.  We plan on
having weekly phone or video--conferences, moderated by Becker and Bloom,
outlining the current state of the project, addressing problems that arise, and
setting goals and expectations for future work.  We will also arrange a
quarterly trip for each postdoc to visit the other's institution, with a
rotating host institution, for direct interaction and camaraderie. We budget for
3 domestic trips per year per institution in this proposal, 2 of which will be
used for these collaborative meetings.  We budget for one domestic and one
international trip per institution for the postdocs to advertise our work at
appropriate conferences in the fields of Astronomy and Computer Science. We
outline the yearly responsibilities for the PIs, the UW postdoc (PD1), and the
USB postdoc (PD2) below.

\medskip {\centerline{\ub{\sc Year 1}}} \smallskip

The first year of the project will require the enumeration of the astrophysical
phenomena we will model in this project (Bloom), the aggregation and calibration
of data on these events from extant data archives (PD1 + Becker), the
development of the statistical infrastructure to build the ST models from these
input data (PD2 + Bloom + Becker), and importantly the staging of the input data
for the Principal Component Analyses (PD1 + PD2).

\medskip {\centerline{\ub{\sc Year 2}}} \smallskip

In the second year we come to the meat of the project, where we must build
viable models from the aggregate input data.  The input data themselves will be
inhomogeneous, meaning we will have to converge on the optimal configuration of
each ST surface dynamically. PD1 will be responsible for understanding how the
input data may effectively be translated into output models ${\cal S}_k$
(surface resolution in $\lambda, t$; number of principal components the data
will support w/Connolly), and PD2 will be responsible for understanding how to
effectively implement a sparse PCA with these data, and how sparsity impacts the
uncertainties of the surface at each $\lambda, t$ bin. Becker will actively
participate in both of these efforts using as reference his understanding of the
SALT--II algorithm and its implementation details.


\medskip {\centerline{\ub{\sc Year 3}}} \smallskip

The final year of the project will include validation of the models, and
incorporation of the models into a classification service.  PD1 will be
responsible for developing a fitting infrastructure that takes as inputs a new
data stream and returns the best--fit parameters from each surface ($x_{ij}$ and
period or redshift).  PD2 will be responsible for taking the results of this
fitting process and returning statistical likelihoods that the event is of each
class using ${\cal D}_k$.  PD2 will also be responsible for incorporating the
overall ST infrastructure into an on--line classification resource that is
available to the astronomical community. PD1 will be responsible for enforcing
validation of the input data to this service, while PD2 will be responsible for
validating the probabilities returned from this fitting process (through
monte--carlo simulations and using sources of vetted data that were not input to
the models). The PIs will examine the results of this training process including
the intrinsic dimensionality of each ${\cal S}_k$ (e.g. is there a component
that allows us to estimate RR Lyrae metallicity from lightcurve shape;
\citealt{1996A&A...312..111J}) and the computational vs. scientific trade--offs
in the calculation of ${\cal D}_k$.

