\documentclass[11pt]{article}

\usepackage{amsmath,amssymb,setspace,geometry,verbatim,graphicx}
\usepackage{url}
\usepackage{fancybox}

\oddsidemargin  0.05in 
\marginparsep 0pt
\topmargin=-0.5in
\textwidth=6.42in
\textheight=9in  
%\parskip = 3pt plus 1pt minus 1pt

 \begin{document}

\subsection*{Postdoc Mentoring}

\section{Postdoctoral Researcher Mentoring Plan}

We plan a comprehensive mentoring program for the postdoctoral scholars
associated with this project. The most robust mentoring starts before
the scholar arrives: by clearly articulating the project goals in
the job position advertisement, the expectations of the postdoc who
eventually accepts the position will be explicit. These expectations---as
well as the expectations of those senior personnel in mentoring positions---will
be reviewed with the scholar upon arrival. An individual development plan (IDP), a planning process that 
identifies both professional development needs and career objectives 
will be prepared. The IDP  includes any suggestions on graduate courses 
that might help the postdoc to actively participate in the cross-disciplinary 
project\footnote{The University of California \textit{Academic Personnel Policy }provides
explicit expectations for mentors of postdoctoral scholars which we
will abide by: ``Faculty mentors are responsible for guiding and monitoring the
advanced training of Postdoctoral Scholars. In that role, faculty
mentors should make clear the goals, objectives, and expectations
of the training program and the responsibilities of Postdoctoral Scholars.
They should regularly and frequently communicate with Postdoctoral
Scholars, provide regular and timely assessments of the Postdoctoral
Scholar's performance, and provide career advice and job placement
assistance.'' (390-6 -- Responsibility). There are similar expectations at U. Washington.}. 
\begin{itemize}
\item \textbf{From the Outset:} The PI and co-PIs will 1) explicitly discuss
the scientific and technical expectations of the postdoc and insure
those are in sync with our expectations, 2) discuss the roles in the
mentor/mentoree relationship, 3) inform the postdoc of avenues for
conflict resolution (following the respective guidelines
and program offices at each institution), and 4) discuss a timeline for progress on the
project.
\item \textbf{Mentoring a Mentor:} The postdoc will be encouraged to take
an active role in co-mentoring (along with the PI and co-PIs) the
graduate students involved in this project. This plays a dual role
in providing valuable, early mentoring experiences in academia for
the postdoc while creating an additional resource for the student
as they progress through their research. We will conduct periodic
conversations about their graduate student mentoring to help the postdoc
reinforce positive results and address difficulties.
\item \textbf{Formal Reviews: }Every 6 months the PI and co-PIs will provide
a formal review to access the research progress, strengths, and areas
needing improvement. 
\item \textbf{Career Laddering}: The PI and co-PIs will informally (but
regularly) address career plans and job search strategies with the
postdoc, helping the postdoc network, obtain invitations to speak
at other universities on the subject of this proposal, and work on
job applications. 
\item \textbf{Encouragement: }Critical to the success of the postdoc (and
this project) will be their continual encouragement to strive for
excellence in their research, allowing ample room for creativity and
independence.\end{itemize}


\end{document}