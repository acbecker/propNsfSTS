\centerline{\bf Collaborative Proposal: A Unifying Description of Astrophysical
Variability for Event Classification} \medskip

Modern time--domain astronomical surveys are designed to uncover the wealth of
temporal astrophysical phenomenology : extrinsic, intrinsic, periodic, and
transient variability of all shapes and amplitudes.  Much of our understanding
of the underlying event physics comes from spectroscopic follow--up, meaning the
rapid and effective classification of new event data is of clear importance.
This is especially true since follow--up resources are saturated even today, and
the ratio of interesting events to our ability to study them will only get worse
with next--generation alert streams. The effective allocation of follow--up
resources requires separating the prosaic from the novel in evolving data
streams, using a holistic classification infrastructure that spans all types of
phenomenology. Since modern surveys will be sampling events sparsely in time,
and irregularly in wavelength, the optimal variability models will be inherently
temporal {\it and} spectral.  {\bf This proposal will build upon the current
state--of--the--art in single--passband event classification to yield
spectral--temporal models of astrophysical variability}.  Models will be built
from the aggregation of photometric and spectroscopic data on variable sources
uncovered by surveys of the past decade, fulfilling their promise as precursor
surveys that may be used to inform future efforts.  As well, theoretical priors
will be used to build models for anticipated classes of new variability.  These
models will be applied to future real--time data streams to classify events as
they themselves evolve, and as our understanding of them grows through
additional photometric or spectroscopic sampling.  We regard these
spectral--temporal surfaces as the logical culmination of event classification
efforts, which have been successfully implemented so far using contextual and
single--passband metrics.

\bigskip \centerline{\ub{\sc Intellectual Merit of the Proposed Activity}}

Our proposal will develop a classification infrastructure based upon the
unifying view of all astrophysical variability as an inherently spectral and
temporal process.  Such a broad attempt has not been made before. However, such
an effort is now possible due to the accumulation of time--domain data over the
past decades, and an increasingly more sophisticated adoption of statistical
techniques by Astronomers. It is also necessary, given the massive volumes of
data expected from next--generation time--domain surveys. The Investigators of
this proposal have an extensive history in the gathering of time--domain
astronomical data, and in their real--time interpretation.  They are directly
connected to past, current, and future time--domain surveys, and are thus
ideally positioned to undertake this effort in a way that is guaranteed to be
impactful.

\bigskip \centerline{\ub{\sc Broader Impacts of the Proposed Activity}}

The derived spectral--temporal models will be provided as a resource to the
astronomical community.  We will also develop an automated classification
resource that accepts data from time--domain event streams, compares these
evolving data to the set of spectral--temporal models, and returns probabilistic
classifications that an event is of a given type.  This resource will provide
value--added information to the survey data, and will help the entire community
to sift through these increasingly more impenetrable event streams. This is an
inherently multidisciplinary effort, requiring astronomical domain knowledge to
aggregate the acquired data and classify known phenomena, and statistical tools
to build the spectral--temporal models and to use them in inference of an
incoming data stream.