\documentclass[11pt]{article}

\usepackage{amsmath,amssymb,setspace,geometry,verbatim,graphicx}
\usepackage{url} \usepackage{fancybox}

\oddsidemargin  0.05in \marginparsep 0pt \topmargin=-0.5in \textwidth=6.42in
\textheight=9in %\parskip = 3pt plus 1pt minus 1pt

\begin{document}

\section*{\label{sec:Data-Management-Plan}Data Management Plan}

\setcounter{page}{1}


\subsection*{Project Data Types}

Project outcomes will include publications of algorithmic development,
spectral--temporal models of astrophysical variability, and a web--based
framework for classification of ongoing survey data.  Testing and verification
datasets will be collected either from open data programs, or through
proprietary access (with no intention to archive 3rd party primary data).

We will conduct all software development on the publicly available open-source
collaboration website {\url{http://github.com}}, where many libraries that serve
as the foundation for this project are already hosted.  The website has become a
leading collaboration platform; it enables distributed users to download code
and contribute back to the project, thanks to a powerful peer--review system.
The underlying version control system behind the website is Git, created to
support the development of the Linux kernel. By design, Git includes -- with
every project download -- its entire development history, ensuring that even if
central repositories on github were to disappear overnight, thousands of
cryptographically verifiable, perfect copies of the project's history exist
across the internet. Hence, there is no single point of failure that could cause
a loss of the project's data; it is also possible to switch to a different
hosting system without loss.

We will, as a safety layer, always host a mirror of all Git repositories and
media files of the project in a central server.  This provides an easy to
locate, trusted copy of the active development sites in case of e.g. temporary
service outages.

We will write software with an eye towards integration in current (PTF) and
future (LSST) real--time classification streams.  This means implementation in
the Python and (optionally) C++ languages.

\subsection*{Data Standards}

The input training data will be stored and consumed in Virtual Astronomical
Observatory\footnote{\url{http://www.usvao.org/}} defined format.  We do not
currently plan on making this collected archive available to the general
community, but adopting these standards will make our API design (and any
unanticipated collaboration) straightforward.  It is unclear if the ST surfaces
are themselves trivially VAO--compatible, but we will make every effort to
represent them as such, to the extent possible.  This would ensure that no
proprietary tools are necessary to view, use, or contribute back to any of the
project outcomes.

\subsection*{Privacy and Intellectual Property Issues}

Our project does not need collection of personally sensitive data; we can thus
make all of our outcomes available under open source licensing terms without any
requirements of confidentiality.


\subsection*{Re--Use and Re--Distribution Licensing Terms}

All outcomes will be made available to the public following the principles set
forth in the Reproducible Research Standard proposed by Stodden
\cite{Stodden-09}:

\begin{itemize}

\item All code we develop for this project will be made available under the
terms of the open source BSD license,
\footnote{\url{http://www.opensource.org/licenses/bsd-license.php} } whenever
possible (i.e., as long as it is completely original code or derived from
similarly licensed code).  In the event that we contribute to projects with
existing licenses that are not compatible with the BSD license, such as LGPL- or
GPL-licensed projects, we will contribute to these projects under the terms of
their own licenses.

\item Educational materials (e.g. documentation) and other media will be
released under the terms of the Creative Commons Attribution License CC~BY,
\footnote{\url{http://creativecommons.org/licenses/by/3.0} } except in cases
where we may reuse materials released under a more restrictive license such as
the Attribution-ShareAlike CC~BY-SA
\footnote{\url{http://creativecommons.org/licenses/by-sa/3.0} } (used by e.g.
Wikipedia). Such materials would be released under the terms of the original
license, in compliance with its original terms.

\item Since US Copyright law prevents the copyright of raw facts, any data
generated as part of our project will be released under the Creative Commons CC0
terms, \footnote{\url{http://creativecommons.org/about/cc0} } i.e., fully
released to the public domain without further copyright claims.

\end{itemize}

\subsection*{Long-Term Archival Plans}

While Git's distributed nature effectively uses the entire internet as a backup
system, as indicated above we will use our server to host a mirror of all of our
materials on the UC Berkeley and University of Washington networks.  We also
expect extant and developing classification streams to uptake our models, which
will ensure they persist (and potentially evolve) in an active environment.

\begin{thebibliography}{99}

\bibitem{Stodden-09} Stodden, V. (2009). \newblock {Enabling Reproducible
Research: Open Licensing For Scientific Innovation}. \newblock
{\textit{International Journal of Communications Law and Policy}, vol. 13}.

\end{thebibliography}




\end{document} 